\begin{center}
\chapter*{Apêndice A - \textit{Strings} de busca empregadas} \label{anex:anexo1}
\end{center}
\addcontentsline{toc}{chapter}{Apêndice A - \textit{Strings} de busca empregadas}

Os dez artigos estudados foram obtidos após buscas no \textit{Google Scholar}, onde foram utilizadas diferentes \textit{strings}, conforme mostra a Tabela abaixo. 

\begin{center}
\scalefont{0.85}
\begin{longtable}{ll}
\caption{\textit{Strings} de busca pesquisadas e artigos encontrados.}\\
\hline
\multicolumn{1}{c}{\textit{String}} & \multicolumn{1}{c}{Artigos} \\ \hline
\endfirsthead

\endhead

\endfoot

\endlastfoot

% \textbf{TESTE}                            & Envia a \textit{string} "TEST\textbackslash n" de volta  \\ \hline
% \textbf{WRD}                        & Responde com o tipo de estação meteorológica \\ \hline
% \textbf{RXCHECK}                        & Responde com o diagnóstico do Console \\ \hline
% \textbf{RXTEST}                       & Muda a tela do console de \textit{"Receiving from"} para tela de dados atuais                                                        \\ \hline
% \textbf{VER}                           & Responde com a data do \textit{firmware}                                                             \\ \hline
% \textbf{RECEIVERS}                    & Responde com a lista das estações que o console "enxerga" \\ \hline
% \textbf{NVER}                       & Responde com a versão do \textit{firmware}                                                             \\ \hline
% \textbf{LOOP}                     & Responde com a quantidade de pacotes especificada a cada 2s        \\ \hline
% \textbf{LPS}                & Responde a cada 2s com a quantidade de pacotes diferentes especificada          \\ \hline
% \textbf{HILOWS}                & Responde com todo os dados de \textit{high/low}                 \\ \hline
% \textbf{PUTRAIN}                      & Seta a quantidade anual de precipitação \\ \hline
% \textbf{PUTET}                 & Seta a quantidade anual de evapotranspiração        \\ \hline
% \textbf{DMP}                 & Faz o \textit{download} de todo o arquivo de memória \\ \hline
% \textbf{DMAFT}                   & Faz o \textit{download} de todo o arquivo de memória após a data especificada \\ \hline
% \textbf{GETEE}                 & Lê toda a memória EEPROM \\ \hline
% \textbf{EEWR}                   & Escreve um \textit{byte} de dados à partir do endereço especificado                                   \\ \hline
% \textbf{EERD}                   & Lê a quantidade de dados especificada iniciando no endereço especificado                                   \\ \hline
% \textbf{EEBWR}                   & Escreve os dados na EEPROM                                    \\ \hline
% \textbf{EEBRD}                   & Lê os dados da EEPROM \\ \hline
% \textbf{CALED}                 & Envia os dados da temperatura e umidade corrente para atribuir à calibração \\ \hline
% \textbf{CALFIX}                   & Atualiza o \textit{display} quando os números de calibração mudam\\ \hline
% \textbf{BAR}                   & Seta os valores da elevação e o \textit{offset} do barômetro quando a localização é alterada                                   \\ \hline
% \textbf{BARDATA}                   & Mostra os valores atuais da calibração do barômetro                                   \\ \hline \\
% \textbf{CLRLOG}                 & Limpa todo o arquivo de dados                                                       \\ \hline
% \textbf{CLRALM}                   & Limpa todos os limiares dos alarmes                                   \\ \hline
% \textbf{CLRCAL}                   & Limpa todos os \textit{offsets} da calibração da temperatura e da umidade \\ \hline
% \textbf{CLRGRA}                   & Limpa o gráfico do console \\ \hline
% \textbf{CLRVAR}                   & Limpa o valor da precipitação ou da evapotranspiração \\ \hline
% \textbf{CLRHIGHS}                   & Limpa todos os valores de pico diários, mensais ou anuais                                   \\ \hline
% \textbf{CLRLOWS}                   & Limpa todos os valores de mínimos diários, mensais ou anuais \\ \hline
% \textbf{CLRBITS}                   & Limpa os \textit{bits} de alarme ativos                                  \\ \hline
% \textbf{CLRDATA}                   & Limpa todos os dados atuais                                   \\ \hline
% \textbf{BAUD}                 & Atribui o valor do \textit{baudrate} do console                                                       \\ \hline
% \textbf{SETTIME}                   & Define a data e a hora do console                                   \\ \hline
% \textbf{GAIN}                   & Define o ganho do receptor de rádio                                   \\ \hline
% \textbf{GETTIME}                   & Retorna a hora e a data atual do console                                   \\ \hline
% \textbf{SETPER}                   & Define o intervalo de arquivamento                                   \\ \hline
% \textbf{STOP}                   & Desabilita a criação dos registros                                   \\ \hline
% \textbf{START}                   & Habilita a criação dos arquivos \\ \hline
% \textbf{NEWSETUP}                   & Reinicia o console após alguma configuração nova                                  \\ \hline
% \textbf{LAMPS}                   & Liga ou desliga as lâmpadas do console \\ \hline

\textit{Mango image processing}				& - \textit{Image processing and analysis techniques for estimating weight} \\
											& \textit{of Chokanan mangoes} \\ 
											& - \textit{A least-squares support vector machine (LS-SVM) based on fractal}  \\ 
											& \textit{analysis and CIElab paramaters for the detection of browning degree} \\ 
											& \textit{on mango (Mangifera indica L.)} \\ \hline

\textit{Mango image processing maturity}	& - \textit{A machine vision-based maturity prediction system for sorting of} \\
											&	\textit{harvested mangoes} \\ \hline

\textit{Mango camera sugar content}			& - \textit{Chokanan mango sweetness determination using HSB color space} \\ \hline

\textit{Mango classification} 				& - \textit{Computer vision system applied to classification of “Manila”} \\
											& 	\textit{mangoes during ripening process} \\ \hline

\textit{Mango HSV}							& - \textit{Mango Fruit Sortation System using Neural Network and Computer} \\
											& 	\textit{Vision} \\ \hline

\textit{Mango RGB}							& - \textit{Image Processing for Mango Ripening Stage Detection: RGB and} \\
											&	\textit{HSV method} \\ 
											& - \textit{Non-Destructive Quality Grading Of Mango (Mangifera Indica L)} \\
											&	\textit{Based On CIELAB Colour Model and Size}  \\
											& - \textit{Determination of Fruit Ripeness Degree of ‘Carabao’ Mango} \\
											&	\textit{(Mangifera indica L.) using Digital Photometry} 	\\
											& - \textit{Determining Sala Mango Qualities with the use of RGB Images} \\
											&	\textit{Captured by a Mobile Phone Camera} \\ \hline

%\label{tab:6}
\end{longtable}
\legend{\textbf{Fonte: } (Autor, 2019).}
\end{center}