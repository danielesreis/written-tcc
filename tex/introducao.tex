%--------------------------------------------------------------------------------------
% Este arquivo contém a sua introdução, objetivos e organização do trabalho
%--------------------------------------------------------------------------------------
\chapter{Introdução}

O Nordeste é uma das regiões de destaque na produção de manga, particularmente nas áreas irrigadas do semiárido, que apresentam boas condições para o desenvolvimento desta fruta (EMBRAPA SEMIÁRIDO, 2010). Em 2015, a região foi responsável por 70\% das áreas colhidas, na qual a Bahia e Pernambuco foram os maiores contribuidores, alcançando as duas primeiras colocações na produção de manga dentre os demais estados (ANUÁRIO BRASILEIRO DE FRUTICULTURA, 2017). Na Bahia e Pernambuco, destacam-se as cidades de Juazeiro (BA) e Petrolina (PE), localizadas no Vale do São Francisco. No ano de 2016, elas foram responsáveis por quase 30\% da produção nacional (PAM, 2017) e por 89\% da exportação de manga no ano de 2014 (BRANCO, 2016).

Para a manutenção da demanda por mangas do Vale do São Francisco, é necessário prezar pela qualidade do produto, sendo este um fator crucial considerado pelo consumidor no momento da compra. Dentre os atributos determinantes para a qualidade da manga, destaca-se sua doçura (SANTOS NETO et al., 2017). Este atributo é representado pelos sólidos solúveis totais (SST), que representam o conteúdo total de açúcares na fruta (SCHMUTZLER; HUCK, 2016). Entretanto, este atributo é normalmente obtido através de um refratômetro\footnote{\label{ftnote:refratometer}Instrumento óptico utilizado para determinar o índice de refração de uma substância (FERRAZ, 2015). Ele pode ser utilizado para monitorar a concentração de açúcar em uma solução, retornando um valor em ºBrix, que corresponde a um grama de açúcar (sacarose) por 100g da solução.}, que quantifica os SST contido no suco extraído, o que exige a destruição das amostras no processo, impossibilitando a comercialização das mesmas. Assim, surge a necessidade de determinar este atributo através de uma metodologia que não destrua as amostras.

Neste contexto, destaca-se a visão computacional, subárea da Inteligência Artificial que dá aos computadores a capacidade de aprendizado e inferência com base em informações visuais (GONZALEZ; WOODS, 2009). Em um sistema desse tipo as imagens digitais são manipuladas através de processamento de imagens, visando a extração de informações significativas sobre as mesmas. Assim, o teor de SST em mangas pode ser obtido a partir das fotos tiradas das frutas, através de um modelo preditivo que relacione este teor com as informações visuais extraídas da manga. Um sistema de visão computacional, além de eliminar a necessidade da destruição das amostras, pode possibilitar a determinação do atributo desejado de forma rápida e precisa.

\section{Motivação}

O trabalho proposto possui potencial para beneficiar a comercialização da manga ao viabilizar a automatização de processos manuais. Assim, a determinação da qualidade das frutas torna-se mais rápida, econômica e consistente (DONIS-GONZÁLEZ et al., 2013). Como a obtenção do teor de sólidos solúveis não exige a destruição das amostras, produtoras \textcolor{red}{e/ou consumidores} de mangas não teriam mais o prejuízo do descarte, além de poderem proporcionar ao mercado consumidor um produto com qualidade assegurada. 

\section{Definição do Problema}

Para a obtenção de uma informação desejada a partir de uma imagem, é necessário cumprir uma sequência de etapas que se inicia na aquisição das imagens das amostras. Segundo Esquef (2002), após a aquisição é realizado o pré-processamento das imagens. Esta etapa é fundamental em um sistema de visão computacional, sendo necessária para remover informações indesejadas, realçar ou atenuar algumas das partes da imagem. Após ser realizado o pré-processamento, é aplicada a segmentação, para que a imagem seja dividida em partes disjuntas, visando sua simplificação (JOSEPH; SINGH, 2014). No contexto deste trabalho, a única região de interesse é a manga; assim, as imagens foram segmentadas visando a separação entre a fruta e o plano de fundo. A penúltima etapa realizada foi a extração de atributos, em que para cada imagem é obtido um conjunto de características que a define. Por fim, estes atributos de cada imagem foram relacionados ao atributo desejado através de um modelo preditivo.

Entretanto, a escolha de técnicas de pré-processamento de imagens e atributos empregados não é realizada de forma trivial e depende da natureza do problema. Logo, para a construção de um bom modelo preditivo de sólidos solúveis totais (SST), diferentes técnicas de pré-processamento foram testadas e, para a determinação dos atributos mais significativos, um estudo investigativo sobre elas foi conduzido. Por fim, a relação entre os atributos e o SST deverá ser modelada através de uma Regressão linear e Random Forest, cujos desempenhos foram avaliados através de métricas apropriadas. Assim, através deste estudo se buscou um profundo entendimento entre os aspectos visuais da manga e sua doçura. 

\section{Objetivos}

\subsection{Objetivo Geral}
O objetivo geral do trabalho consiste na identificação de quais atributos de uma imagem da manga devem ser considerados para a predição dos sólidos solúveis. 

\subsection{Objetivos específicos}

\begin{itemize}
	\item Determinar as técnicas de pré-processamento de imagens mais adequadas para o problema;
    \item Realizar um estudo investigativo quanto aos atributos através de modelos de Regressão linear e Random Forest;
    \item \textcolor{red}{Correlacionar as imagens com os atributos físico-químicos das mangas.}
\end{itemize}

\section{Organização do trabalho}

O trabalho possui, além da Introdução, os capítulos de Fundamentação teórica, Materiais e métodos, Resultados e Conclusão. Na fundamentação é discorrido sobre todos os tópicos necessários para o entendimento do trabalho e os artigos selecionados para o estudo. No capítulo de materiais e métodos, é explicada a metodologia a ser aplicada, desde a coleta das mangas até a construção dos modelos preditivos. Os resultados obtidos no trabalho são mostrados no capítulo seguinte e, por fim, no último capítulo é feita uma conclusão sobre o estudo conduzido, assim como potenciais trabalhos futuros.
