\chapter{Referências}

\noindent ALVES, E. D. L.; VECCHIA, F. A. S. Análise de diferentes métodos de interpolação para a precipitação pluvial no Estado de Goiás. \textit{Acta Scientiarum. Human and Social Sciences}, v. 33, n. 2, 2011. Disponível em: <http://www.redalyc.org>. Acesso em: 01 ago. 2018.
\\

\noindent ABARRA, M. S. J. et al. Determination of Fruit Ripeness Degree of ‘Carabao’ Mango (Mangifera indica L.) using Digital Photometry. \textit{Philippine Journal of Science}, v. 147, n. 2, p. 249-253, 2018. Disponível em: <http://philjournalsci.dost.gov.ph/ >. Acesso em: 27 jun. 2018.
\\

\noindent ANUÁRIO BRASILEIRO DE FRUTICULTURA. Santa Cruz do Sul: Editora Gazeta, 2017. Disponível em: <http://www.editoragazeta.com.br>. Acesso em: 18 ago. 2018.
\\

\noindent ARCO, J. E. et al. Digital image analysis for automatic enumeration of malaria parasites using morphological operations. \textit{Expert Systems with Applications}, v. 42, n. 6, p. 3041-3047, 2015. Disponível em: <www.sciencedirect.com>. Acesso em: 29 jul. 2018.
\\

\noindent BEDI, S. S.; KHANDELWAL, Rati. Various image enhancement techniques-a critical review. \textit{International Journal of Advanced Research in Computer and Communication Engineering}, v. 2, n. 3, 2013. Disponível em: <https://pdfs.semanticscholar.org>. Acesso em: 29 jul. 2018.
\\

\noindent BRANCO, Danyelle Karine Santos et al. Comportamento das exportações de manga do Vale Submédio São Francisco: uma abordagem a partir de vetores autorregressivos. \textit{Revista Econômica do Nordeste}, v. 47, n. 4, p. 29-37, 2016. Disponível em: <https://www.bnb.gov.br>. Acesso em: 04 set. 2018.
\\

\noindent BREIMAN, L. Random forests. \textit{Machine learning}, v. 45, n. 1, p. 5-32, 2001.
\\

\noindent CAMASTRA, F. et al. A fuzzy decision system for genetically modified plant environmental risk assessment using Mamdani inference. \textit{Expert Systems with Applications}, v. 42, n. 3, p. 1710-1716, 2015.
\\

\noindent DA SILVA, L. A.; PERES, S. M.; BOSCARIOLI, C. \textit{Introdução à mineração de dados: com aplicações em R}. Elsevier Brasil, 2017.
\\

\noindent DE ABREU, A. L. E. \textit{Bootstrap e modelos de Support Vector Machine-SVM}. 2016. 117f. Dissertação (Doutorado em Métodos Numéricos) – Universidade Federal do Paraná, Curitiba, 2016. Disponível em: <www.acervodigital.ufpr.br>. Acesso em: 16 jul. 2018.
\\

\noindent DONIS-GONZÁLEZ, I. R. et al. Assessment of chestnut (Castanea spp.) slice quality using color images. \textit{Journal of Food Engineering}, v. 115, n. 3, p. 407-414, 2013. Disponível em: <www.sciencedirect.com>. Acesso em: 17 jun. 2018.
\\

\noindent DOMINGO, D. L. et al. Digital photometric method for determining degree of harvest maturity and ripeness of ‘Sinta’papaya (Carica papaya L.) fruits. \textit{The Philippine Agricultural Scientist}, v. 95, n. 3, 2013. Disponível em: <https://journals.uplb.edu.ph>. Acesso em: 18 jul. 2018.
\\

\noindent DORJ, U.; LEE, M.; YUN, S.. An yield estimation in citrus orchards via fruit detection and counting using image processing. \textit{Computers and Electronics in Agriculture}, v. 140, p. 103-112, 2017. Disponível em: <www.sciencedirect.com>. Acesso em: 27 jul. 2018.
\\

\noindent DOS SANTOS NETO, J. P. et al. Determination of ‘Palmer’mango maturity indices using portable near infrared (VIS-NIR) spectrometer. \textit{Postharvest Biology and Technology}, v. 130, p. 75-80, 2017. Disponível em: <www.sciencedirect.com>. Acesso em: 02 jul. 2018. 
\\

\noindent DUARTE FILHO, J. et al. Aspectos do florescimento e técnicas empregadas objetivando a produção precoce em morangueiros. \textit{Informe Agropecuário}, v. 20, n. 198, p. 30-35, 1999. 
\\

\noindent EMBRAPA SEMIÁRIDO. \textit{Cultivo da mangueira}. Petrolina, 2002. Disponível em: \\ <https://ainfo.cnptia.embrapa.br>. Acesso em: 08 ago. 2018.
\\

\noindent ENVI - Guia do ENVI em Português. Sulsoft, 2000. Diponível em: <www.sulsoft.com.br>. Acesso em: 17 jul. 2018.
\\

\noindent ESQUEF, A. \textit{Técnicas de entropia em processamento de imagens}. RIO DE JANEIRO: CENTRO BRASILEIRO DE PESQUISAS FÍSICAS, 2002.
\\

\noindent FERRAZ, M. N. \textit{Uso da espectrometria para investigação da qualidade da cana-de-açúcar em campo.}2015. 148 p. Dissertação (Mestrado em Engenharia de Sistemas Agrícolas) - Universidade de São Paulo, Piracicaba, 2015.
\\

\noindent FONSECA, J. J. S. Metodologia da pesquisa científica. Fortaleza: UEC, 2002. Apostila.
\\

\noindent FRIEDMAN, Jerome; HASTIE, Trevor; TIBSHIRANI, Robert. \textit{The elements of statistical learning}. New York, NY, USA:: Springer series in statistics, 2001.
\\

\noindent GAMA, João et al. \textit{Extração de conhecimento de dados: data mining}. 2015.
\\

\noindent GIL, A. C. Como elaborar projetos de pesquisa. 4. ed. São Paulo: Atlas, 2007.
\\

\noindent GLOBAL IMAGE THRESHOLDING USING OTSU'S METHOD. MathWorks Inc. Disponível em: <https://www.mathworks.com/help/images/ref/graythresh.html>. Acesso em: 17 feb. 2019.
\\

\noindent GONZALEZ, R. C.; WOODS, R. C. \textit{Processamento digital de imagens}. Pearson Educación, 2009.
\\

\noindent HAYKIN, S. \textit{Redes neurais: princípios e prática}. Porto Alegre: Bookman, 2001. 900 p.
\\

\noindent HEMALATHA, G.; SUMATHI, C. P. Preprocessing techniques of facial image with Median and Gabor filters. In: \textit{Information Communication and Embedded Systems (ICICES), 2016 International Conference on}. IEEE, 2016. p. 1-6. Disponível em: <https://ieeexplore.ieee.org>. Acesso em: 29 jul. 2018.
\\

\noindent HUTENGS, C.; VOHLAND, M. Downscaling land surface temperatures at regional scales with random forest regression. \textit{Remote Sensing of Environment}, v. 178, p. 127-141, 2016. Disponível em: <www.sciencedirect.com>. Acesso em: 02 ago. 2018.
\\

\noindent IAL - INSTITUTO ADOLFO LUTZ. \textit{Métodos físico-químicos para análise de alimentos}. São Paulo, 2008. 1020 p.
\\

\noindent IMAGE THRESHOLDING. OpenCV. Disponível em: <https://docs.opencv.org/3.4/d7/d4d/tutorial\_py\_thresholding.html>. Acesso em: 17 feb. 2019.
\\

\noindent JAMES, Gareth et al. \textit{An introduction to statistical learning}. New York: springer, 2013.
\\

\noindent JATMIKA, S.; PURNAMASARI, D.. Rancang Bangun Alat Pendeteksi Kematangan Buah Apel dengan Menggunakan Metode Image Processing Berdasarkan Komposisi WARNA. \textit{Jurnal Ilmiah Teknologi Informasi Asia}, v. 8, n. 1, p. 51-58, 2014. Disponível em: <https://jurnal.stmikasia.ac.id>. Acesso em: 18 jul. 2018.
\\

\noindent JHA, S. N.; KINGSLY, A. R. P.; CHOPRA, Sangeeta. Physical and mechanical properties of mango during growth and storage for determination of maturity. \textit{Journal of Food engineering}, v. 72, n. 1, p. 73-76, 2006. Disponível em: <www.sciencedirect.com>. Acesso em: 02 jul. 2018.
\\

\noindent JOSEPH, R. P.; SINGH, C. S.; MANIKANDAN, M. Brain tumor MRI image segmentation and detection in image processing. International \textit{Journal of Research in Engineering and Technology}, v. 3, n. 1, p. 1-5, 2014. Disponível em: <https://s3.amazonaws.com>. Acesso em: 18 jul. 2018.
\\

\noindent KAUR, A.; KRANTHI, B. V. Comparison between YCbCr color space and CIELab color space for skin color segmentation. \textit{IJAIS}, v. 3, n. 4, p. 30-33, 2012. Disponível em: <https://pdfs.semanticscholar.org>. Acesso em: 29 jul. 2018.
\\

\noindent KHAIRUNNIZA-BEJO, S.; KAMARUDIN, S. Chokanan mango sweetness determination using hsb color space. In: \textit{Computational Intelligence, Modelling and Simulation (CIMSiM), 2011 Third International Conference on}. IEEE, 2011. p. 216-221. Disponível em: <www.ieeexplore.ieee.org>. Acesso em: 13 jun. 2018.
\\

\noindent MIHALOVIC, M. Performance comparison of multiple discriminant analysis and logit models in bankruptcy prediction. \textit{Economics \& Sociology}, v. 9, n. 4, p. 101, 2016. Disponível em: <http://www.economics-sociology.eu>. Acesso em: 17 jul. 2018.
\\

\noindent MORPHOLOGICAL TRANSFORMATIONS. OpenCV-Python Tutorials. Disponível em: <https://opencv-python-tutroals.readthedocs.io/en/latest/py\_tutorials/py\_imgproc/py\_morphological\_ops/py\_morphological\_ops.html>. Acesso em: 17 jul. 2018.
\\

\noindent 2-D MEDIAN FILTERING. MathWorks Inc. Disponível em: <https://www.mathworks.com/help/images/ref/medfilt2.html>. Acesso em: 17 jul. 2018.
\\

\noindent NADAFZADEH, M.; MEHDIZADEH, S. A.; SOLTANIKAZEMI, M. Development of computer vision system to predict peroxidase and polyphenol oxidase enzymes to evaluate the process of banana peel browning using genetic programming modeling. \textit{Scientia Horticulturae}, v. 231, p. 201-209, 2018. Disponível em: <www.sciencedirect.com>. Acesso em: 27 jul. 2018.
\\

\noindent NANDI, C. S.; TUDU, B.; KOLEY, C. A machine vision-based maturity prediction system for sorting of harvested mangoes. \textit{IEEE Transactions on Instrumentation and measurement}, v. 63, n. 7, p. 1722-1730, 2014. Disponível em: <https://ieeexplore.ieee.org>. Acesso em: 12 jun. 2018.
\\

\noindent NETO, F. P. L. Novas opções de variedades de mangueira e as vantagens competitivas. In: \textit{Embrapa Semiárido-Artigo em anais de congresso (ALICE). In: FEIRA NACIONAL DA AGRICULTURA IRRIGADA-FENAGRI, 20.}, 2009. Disponível em: <www.alice.cnptia.embrapa.br>. Acesso em: 31 jul. 2018.
\\

\noindent PAM – \textit{Produção Agrícola Municipal}. Disponível em: <https://sidra.ibge.gov.br/pesquisa/pam>. Acesso em: 08 ago. 2018.
\\

\noindent PANDEY, R.; GAMIT, N.; NAIK, S. Non-destructive quality grading of mango (Mangifera Indica L) based on CIELab colour model and size. In: \textit{Advanced Communication Control and Computing Technologies (ICACCCT), 2014 International Conference on}. IEEE, 2014. p. 1246-1251. Disponível em: <http://ieeexplore.ieee.org>. Acesso em: 27 jun. 2018.
\\

\noindent PERMADI, Y. et al. Aplikasi Pengolahan Citra Untuk Identifikasi Kematangan Mentimun Berdasarkan Tekstur Kulit Buah Menggunakan Metode Ekstraksi Ciri Statistik. \textit{Jurnal Informatika}, v. 9, n. 1, 2015. Disponível em: <http://www.journal.uad.ac.id>. Acesso em: 18 jul. 2018.
\\

\noindent PRAVEEN, K. S. et al. Implementation Of Image Sharpening And Smoothing Using Filters. \textit{International Journal of Scientific Engineering and Applied Science}, v. 2, n. 1, p. 7-14, 2016. Disponível em: <http://ijseas.com>. Acesso em: 29 jul. 2018.
\\

\noindent RUSSELL, Stuart J.; NORVIG, Peter. \textit{Artificial intelligence: a modern approach}. Malaysia; Pearson Education Limited,, 2016.
\\

\noindent SALUNKHE, R., P.; PATIL, A. A. Image processing for mango ripening stage detection: RGB and HSV method. In: \textit{Image Information Processing (ICIIP), 2015 Third International Conference on}. IEEE, 2015. p. 362-365. Disponível em: <https://ieeexplore.ieee.org/>. Acesso em: 13 jun. 2018.
\\

\noindent SCHMUTZLER, M.; HUCK, C. W. Simultaneous detection of total antioxidant capacity and total soluble solids content by Fourier transform near-infrared (FT-NIR) spectroscopy: a quick and sensitive method for on-site analyses of apples. \textit{Food Control}, v. 66, p. 27-37, 2016.
\\

\noindent SHAIK, K. B. et al. Comparative study of skin color detection and segmentation in HSV and YCbCr color space. \textit{Procedia Computer Science}, v. 57, p. 41-48, 2015. Disponível em: <www.sciencedirect.com >. Acesso em: 17 jul. 2018.
\\

\noindent SONKA, M.; HLAVAC, V.; BOYLE, R. \textit{Image processing, analysis, and machine vision}. Cengage Learning, 2014.
\\

\noindent SRIRAAM, N. Correlation dimension based lossless compression of EEG signals. Biomedical Signal Processing and Control, v. 7, n. 4, p. 379-388, 2012.
\\

\noindent TEOH, C. C.; SYAIFUDIN, A. R. M. Image processing and analysis techniques for estimating weight of Chokanan mangoes. \textit{Journal of Tropical Agriculture and Food Science}, v. 35, n. 1, p. 183, 2007. Disponível em: <http://ejtafs.mardi.gov.my/jtafs/35-1/Chokanan\%20mangoes.pdf>. Acesso em: 11 jun. 2018.
\\

\noindent VÉLEZ-RIVERA, N. et al. Computer vision system applied to classification of “Manila” mangoes during ripening process. \textit{Food and bioprocess technology}, v. 7, n. 4, p. 1183-1194, 2014. Disponível em: <https://link.springer.com/article/10.1007/s11947-013-1142-4>. Acesso em: 11 jun. 2018.
\\

\noindent YADAV, S.; JAIN, C.; CHUGH, A.. Evaluation of image deblurring techniques. \textit{Evaluation}, v. 139, n. 12, 2016. Disponível em: <www.pdfs.semanticscholar.org >. Acesso em: 16 jul. 2018.
\\

\noindent YAHAYA, O. K. M. et al. Determining Sala mango qualities with the use of RGB images captured by a mobile phone camera. In: \textit{AIP Conference Proceedings}. AIP Publishing, 2015. p. 060003. Disponível em: <http://aip.scitation.org>. Acesso em: 27 jun. 2018.
\\

\noindent YOSSY, E. H. et al. Mango Fruit Sortation System using Neural Network and Computer Vis
ion. \textit{Procedia Computer Science}, v. 116, p. 596-603, 2017. Disponível em: <www.sciencedirect.com>. Acesso em: 13 jun. 2018.
\\

\noindent ZHENG, H.; LU, H. A least-squares support vector machine (LS-SVM) based on fractal analysis and CIELab parameters for the detection of browning degree on mango (Mangifera indica L.). \textit{Computers and Electronics in Agriculture}, v. 83, p. 47-51, 2012. Disponível em: <www.sciencedirect.com>. Acesso em 12 jun. 2018.